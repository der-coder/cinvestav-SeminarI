\documentclass[a4paper,12pt]{article}
%\documentclass[a4paper,12pt]{scrartcl}

\usepackage{xltxtra}

\input{../preamble.tex}

\usepackage[spanish]{babel}

% \setromanfont[Mapping=tex-text]{Linux Libertine O}
% \setsansfont[Mapping=tex-text]{DejaVu Sans}
% \setmonofont[Mapping=tex-text]{DejaVu Sans Mono}

\title{La importancia de Mecatrónica en Manufactura}
\author{Isaac Ayala Lozano}
\date{\today}

\begin{document}
\maketitle

La importancia de Mecatrónica en Manufactura es muy grande.
El impacto que ésta ha tenido en todos los niveles de Manufactura puede ser descrita de acuerdo a las múltiples definiciones de Mecatrónica y la relación de cada una con Manufactura.
La Mecatrónica puede ser definida como un concepto tecnológico, el cuál la describe como una evolución de la innovación tecnológica orientada a automatización.
Puede también ser descrita como una metodología de trabajo en la que se emplea la tecnología para la creación de nueva tecnología y organiza a las personas en grupos de trabajo multidisciplinarios.
Finalmente, puede ser definida de acuerdo a su implementación: Procesos y Sistemas de Manufactura.
% Según el área, la Mecatrónica puede ser asociada a
% \emph{Procesos y Sistemas de Manufactura}, \emph{Ingeniería Biomédica}, \emph{Nanotecnología} y \emph{Robótica}.

Como concepto tecnológico, Mecatrónica en Manufactura se refiere a los avances tecnológicos que se han tenido en el diseño y creación de productos.
En la búsqueda de nuevas tecnologías fue posible implementar sistemas electrónicos como substitutos de sistemas mecánicos en líneas de producción.
Un gran ejemplo de esto son los sistemas de control donde sistemas mecánicos como el regulador centrífugo fueron reemplazados por circuitos electrónicos capaces de replicar su funcionamiento.
La migración de sistemas mecánicos a sistemas electrónicos trajo consigo grandes ventajas como la reducción de espacio empleado para estos sistemas, además de una mayor flexibilidad para modificar el sistema existente.

Como metodología de trabajo, la Mecatrónica empujó al área de Manufactura a mejorar considerablemente tanto la forma de trabajo de las personas como los procesos de producción y diseño.
Estas mejoras han conducido a la producción de objetos con un costo mucho menor y que cumplan con estándares de calidad superiores a aquellos que existían antes.
Por parte del uso de tecnología, la implementación de simuladores numéricos en el diseño ha reducido el costo y tiempo necesarios para llevar un producto al mercado.
A su vez, las metodologías de trabajo cambiaron para permitir a equipos multidisciplinarios trabajar de manera mucho más ágil.

Como área de aplicación, la Mecatrónica se ha enfocado en automatizar los procesos y sistemas de manufactura.
Esto ha sucedido con la implementación de sistemas electrónicos programables, brindando gran flexibilidad para la fabricación de objetos.
Además el uso de nueva tecnología ha permitido hacer más rápidos y eficientes a los procesos de manufactura como soldadura, estampado, pintura, traslado de material y unión de piezas.
El uso de tecnología en este caso puede referirse al uso de nuevos herramentales y robots para cada tarea en el proceso, pero también considera los sistemas de sensores, actuadores y de control empleados para la producción de bienes comerciales.

La Mecatrónica ha cambiado significativamente al área de Manufactura.
Ha permitido incrementar el número de productos producidos, así como su variedad y calidad.
Ha traído grandes reducciones de costo tanto a consumidores como productores debido a la optimización del uso de tiempo, material y recursos en los procesos de fabricación.
Ha cambiado también la forma de trabajo de las personas involucradas en el diseño y fabricación de productos con el uso de nuevas tecnologías y estrategias de organización.

La importacia de la Mecatrónica en Manufactura es inmensa.
El área ha sufrido cambios benéficos debido a ella, tanto en procesos y sistemas de producción como en los bienes fabricados.
La Mecatrónica en Manufactura es importante porque ha resultado en productos de buena calidad y bajo costo, cualidades que la Manufactura sin Mecatrónica no sería capaz de proveer al mismo nivel.

% \printbibliography

% \newpage
% \pagebreak
% \appendix
% \section{Octave Code}
% \lstinputlisting[language=Matlab]{<filename>.m}

\end{document}
