\documentclass[a4paper,12pt]{article}
%\documentclass[a4paper,12pt]{scrartcl}

\usepackage{xltxtra}

\input{../preamble.tex}

\usepackage[spanish]{babel}

% \setromanfont[Mapping=tex-text]{Linux Libertine O}
% \setsansfont[Mapping=tex-text]{DejaVu Sans}
% \setmonofont[Mapping=tex-text]{DejaVu Sans Mono}

\title{Importancia de la tecnología informática en la tecnología moderna}
\author{Isaac Ayala Lozano}
\date{2020-03-20}

\begin{document}
\maketitle

La necesidad ha sido siempre la razón del progreso humano.
Las sociedades humanas constantemente desarrollan soluciones para satisfacer las necesidades de su época.
Al estudiar la historia de la humanidad, es posible clasificar las necesidades humana en dos categorías:
 las necesidades que surgen en el ser humano sin importar la época y aquellas que varían de acuerdo a las condiciones del medio en que las personas se desarrollan.

La primer categoría comprende aquellas necesidades que son esenciales para el ser humano, como satisfacer su hambre o poseer una buena salud para efectuar sus actividades.
Estas necesidades no dependen de la época o la sociedad, ya que son parte de los requisitos que cualquier organismo vivo debe cumplir para continuar con su existencia.
En contraste, las necesidades que pertenecen a la segunda categoría son alteradas por el tiempo y las sociedades.
Un ejemplo de esto es la necesidad de un medio de transporte personal para efectuar un oficio como el servicio de entrega de comida a domicilio.
Esta necesidad no es requerida por todos y surge del estilo de vida actual de múltiples sociedades modernas como la sociedad estadounidense o japonesa.

Las necesidades de la sociedad y de los individuos dan paso al avance tecnológico.
El cual puede ser definido como la aplicación del conocimiento de múltiples áreas para la creación de herramientas o estrategias que satisfagan necesidades.
La sociedad busca siempre satisfacer sus necesidades, y a pesar de existir ya algo que pueda satisfacerlas es posible hallar nuevas soluciones a un mismo problema o necesidad.
Dicho esto, puede que cada sociedad desarrolle una respuesta diferente para atender a la misma necesidad.
Considérese el sector energético como ejemplo, habiendo la necesidad de desarrollar medios para obtener energía de manera sustentable se han obtenido múltiples opciones.
Sistemas para la generación de energía proveniente de la luz solar, del movimiento del aire y el océano, así como del calor del planeta, son todos respuestas válidas para atender esta misma necesidad.

La generación de nuevas y mejores maneras de satisfacer una necesidad o resolver un problema son el fundamento del avance tecnológico.
Considérese el transistor como un ejemplo de esto.
El transistor es un dispositivo que permite la amplificación y cambio de señales electrónicas.
Habiendo surgido en Bell Labs en la década de 1940, el transistor dio pasó a un desarrollo masivo de la electrónica e incluso de la sociedad.
Su implementación en un sinfín de aplicaciones como electrónica de consumo, equipo médico y militar, así como su uso en las primeras computadoras revolucionó por completo la vida humana.
Teniendo ya el primer modelo funcional no fue un impedimento para generar mejoras del diseño o incluso el uso de nuevos conceptos para su funcionamiento.
El uso de nuevos conceptos incluye el desarrollo de los diferentes tipos de transistores como lo son los transistoresde tipo Metal-Oxide-Semiconductor Field Effect Transistor (MOSFET), de tipo Bipolar Juntion Transistor (BJT), y del tipo Junction Field-Effect Transistor (JFET).

El desarrollo de estas distintas implementaciones del concepto tuvieron un gran impacto en el desarrollo tecnológico de la humanidad.
Su uso como bloque de construcción de circuitos eléctricos permitió el desarrollo de otras tecnologías como el amplificador operacional.
Éste permite manipular señales eléctricas para realizar operaciones aritméticas, además de derivar e integrar.
Estas nuevas herramientas abrieron paso al uso de computadoras con componentes electrónicos.

El progreso tecnológico que surge de estos eventos continúa repercutiendo en la vida diaria.
Se cuenta ahora con computadoras personales portátiles, empleando además nuevos dispositivos electrónicos que la sociedad actual considera esenciales para su funcionamiento.
Una muestra de esto son los teléfonos celulares y la infraestructura necesaria para realizar todas sus funciones de comunicación, la adquisición y envío de datos, el rastreo de señales y la ejecución de programas a la solicitud del usuario.

Si se observan los eventos de un día cualquiera en la era moderna, la cantidad de cambios que ésta tiene con un día normal de otra era es increíble.
Si se compara incluso con un día en los primeros años del milenio, los cambios continuan siendo impresionantes.
La presencia casi ubicua del teléfono celular es uno de los factores más relevantes.
Este ha reemplazado un sinfín de dispositivos mediante el desarrollo de nuevas aplicaciones para el mismo hardware.
Atualmente, una persona promedio no tiene necesidad de adquirir un despertador, o una cámara fotográfica, e incluso una computadora.
Las funciones más importantes de estos dispositivos ya están disponibles en un teléfono moderno.
De este ejemplo se nota el hecho de que los avances tecnológicos han permitido reducir el número de herramientas que una persona requiere al mínimo posible en ciertas situaciones.

El desarrollo actual de la tecnología se ve impactado casi en su totalidad por la tecnología informática.
Gran parte de los dispositivos actuales hacen uso de los desarrollos recientes de esta área.
Dispositvos inteligentes requieren de medios de almacenaje, sensores, e interfaces para interactuar con su entorno.
Todos éstos son resultados de la tecnología informática: discos duros para almacenaje de información, software para controlar dispositivos, y el desarrollo de nuevas interfaces para que las personas puedan controlarlos.

El constante avance de la tecnología informática ha permitido el mejoramiento de procesos, así como una mejora considerable en el funcionamiento de máquinas.
Esto ha impactado fuertemente en la industria, ya que ahora se cuenta con nuevos niveles de calidad en los productos que antes no serían posibles como la precisión del proceso de ensamble de componentes con aplicación de fuerza y torque específicos.

Es evidente que la tecnología informática es de suma importancia en la tecnología moderna, como se ha mostrado ya en este texto.
El empuje que ésta tiene sobre el progreso tecnológico es indiscutible, ya que son mejoras en la tecnología informática lo que permite el desarrollo de nuevas aplicaciones que en eras pasadas serían consideradas imposibles.


%%%%%%%%%%%%%%%



\end{document}
