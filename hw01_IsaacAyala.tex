\documentclass[a4paper,12pt]{article}
%\documentclass[a4paper,12pt]{scrartcl}

\usepackage{xltxtra}

\input{../preamble.tex}

\usepackage[spanish]{babel}

% \setromanfont[Mapping=tex-text]{Linux Libertine O}
% \setsansfont[Mapping=tex-text]{DejaVu Sans}
% \setmonofont[Mapping=tex-text]{DejaVu Sans Mono}

\title{Sobre la tecnología informática y la evolución de la tecnología}
\author{Isaac Ayala Lozano}
\date{2020-03-06}

\begin{document}
\maketitle

La necesidad ha sido siempre la razón del progreso humano.
Las sociedades humanas constantemente desarrollan soluciones para satisfacer las necesidades de la época.
Al estudiar la historia de la humanidad, uno es capaz de
encontrar dos grupos de necesidades: las necesidades recurrentes sin importar la época y aquellas que surgen de las condiciones del medio en que el ser humano se desarrolla.\\

El primer grupo está constituido por temas esenciales para el ser humano, como la alimentación, la salud y la protección.
El segundo grupo es mucho más diverso.
Cada sociedad define sus propias necesidades, llevando así a diferentes desarollos tecnológicos provenientes de distintas culturas que existen en una misma era.\\

La era actual no es diferente de ello en este aspecto.
Siendo esta era una era de informática y globalización,
las necesidades de las sociedades se enfocan en dichos aspectos.
¿Cómo acceder a la información del mundo de manera segura?
¿Qué uso puede tener tal tecnología?
¿De qué manera se puede emplear la información para resolver un problema?
Preguntas como éstas dan indicios del enfoque actual de la humanidad.\\

El acceso a la información se ha vuelto vital para los individuos impactados por la globalización.
Esta necesidad da lugar a requisitos que la tecnología actual debe de cumplir.
Disponibilidad, cobertura, velocidad, respaldos, almacenaje, adquisición de datos; conceptos como éstos guían el desarrollo actual de la tecnología.
El manejo de información, por ejemplo, ha llevado al desarrollo de nuevas tecnologías para la transmisión de información de manera más rápida, en canales de comunicación seguros y con la posibilidad de almacenar cada vez una mayor cantidad de datos.\\

Otro caso relevante es el microprocesador.
La necesidad de obtener velocidades de procesamiento cada vez más altas llevó a la industria a refinar los procesos de manufactura hasta el límite.
Siendo el límite el tamaño mínimo de cada transistor en el microporcesador.
Habiendo llegado ya a dicha limitante, la industria ha recurrido a desarrollar nuevas alternativas para alcanzar velocidades todavía más altas de manera más eficiente.
Ejemplo de esto es el estudio de la luz como un nuevo medio de transmisión de datos con el objetivo de disminuir la generación de calor en los microprocesadores actuales, aunado a la densidad de transistores incluidos actualmente en ellos.\\

La disponibilidad de información y la adquisición de datos han jugado también un papel primordial en el desarrollo de la tecnología.
La posibilidad de conocer de manera detallada el comportamiento de fenómenos y objetos ha permitido el desarrollo de nuevos equipos con especificaciones y capacidades superiores a sus precursores.
A su vez, estos equipos han permitido el refinamiento de procesos de manufactura y la calidad de fabricación de productos.
Estos últimos permiten a su vez la creación de maquinaria aún más especializada.
Este ciclo de mejoramiento es posible gracias al desarrollo de la tecnología informática y el incremento en los requisitos que esta misma impone en los avances de la tecnología en general.\\

Ejemplos como éstos son muestras de cómo la tecnología informática ha ocasionado la evolución actual de la tecnología.
Teniendo ahora a una humanidad con una creciente necesidad de acceder a la información, esta tendencia se verá fortalecida más y más con el paso del tiempo.
Además, avances en otras áreas del conocimiento abren a su vez nuevas puertas para el desarrollo de tecnología.
De igual manera estos posibles nuevos desarrollos impulsarán mucho más a la tecnología informática en formas no previsibles.


% \printbibliography

\end{document}
